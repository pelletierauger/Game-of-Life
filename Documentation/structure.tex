%!TEX program = xelatex
\documentclass[12pt]{article}
\usepackage{fontspec}
\usepackage[top=1.25in, bottom=1.25in, left=1.25in, right=1.25in]{geometry}
\usepackage{graphicx}
\usepackage{polyglossia}
\usepackage{amsmath}
\setmainlanguage{french}
\usepackage{inconsolata}
\usepackage[T1]{fontenc}
\usepackage{listings, xcolor}
\DeclareTextCommandDefault{\nobreakspace}{\leavevmode\nobreak\ }
\makeatletter
\g@addto@macro\@floatboxreset{\centering}
\makeatother

\usepackage{color}
\definecolor{grey}{rgb}{0.95,0.95,0.95}
\definecolor{lightgray}{rgb}{.9,.9,.9}
\definecolor{darkgray}{rgb}{.4,.4,.4}
\definecolor{darkblue}{rgb}{0, 0.3, 0.82}
\definecolor{darkorange}{rgb}{0.8, 0.4, 0}
\definecolor{darkgreen}{rgb}{0, 0.55, 0.2}
\definecolor{darkpurple}{rgb}{0, 0.6, 0.6}
\definecolor{p5pink}{rgb}{0.85, 0, 0.5}
\setlength\marginparsep{10pt}
\setlength\marginparwidth{60pt}
% \setlength{\parskip}{1em}

\lstdefinelanguage{JavaScript}{
    keywords={break, case, catch, const, continue, debugger, default, delete, do, else, false, finally, for, function, if, in, instanceof, let, new, null, return, switch, this, throw, true, try, typeof, var, void, while, with},
    keywordstyle=\lst@ifdisplaystyle\color{darkblue}\ttfamily\fi,
    ndkeywords={class, export, boolean, throw, implements, import, this},
    ndkeywordstyle=\color{darkgray}\ttfamily,
    identifierstyle=\color{black},
    sensitive=false,
    comment=[l]{//},
    morecomment=[s]{/*}{*/},
    commentstyle=\color{darkgray}\ttfamily,
    stringstyle=\color{darkorange}\ttfamily,
    morestring=[b]',
    morestring=[b]",
    keepspaces=true,
    showstringspaces=false,
    classoffset=1,
    morekeywords={background,blendMode,createVector,createCanvas,dist,ellipse,fill,
    lerp,line,map,rect,translate,
    abs,acos,asin,atan,atan2,cos,floor,round,sin,stroke,strokeWeight,tan,pow},
    keywordstyle=\lst@ifdisplaystyle\color{p5pink}\ttfamily\fi
}

\lstset{backgroundcolor=\color{grey},
    numbers=left,
    columns=fullflexible,
    basicstyle=\ttfamily\lst@ifdisplaystyle\fontsize{10}{12}\fi,
    language=JavaScript,
    numberstyle=\color{darkgray},
    xleftmargin=12pt,xrightmargin=12pt,
    aboveskip=12pt,belowskip=12pt,
    frame=tlbr,framesep=12pt,framerule=0pt,
    numbers=none,
    lineskip={1pt}
}

\lstset{literate=%
    *{0}{{{\color{darkgreen}0}}}1
    {1}{{{\color{darkgreen}1}}}1
    {2}{{{\color{darkgreen}2}}}1
    {3}{{{\color{darkgreen}3}}}1
    {4}{{{\color{darkgreen}4}}}1
    {5}{{{\color{darkgreen}5}}}1
    {6}{{{\color{darkgreen}6}}}1
    {7}{{{\color{darkgreen}7}}}1
    {8}{{{\color{darkgreen}8}}}1
    {9}{{{\color{darkgreen}9}}}1
}
\def\inline{\lstinline[basicstyle=\ttfamily, keywords={},keywordstyle={}]}


\author{Guillaume Pelletier-Auger}
\title{Expérimentations animées\protect\\avec le \textit{jeu de la vie} de John Conway}
\date{10 décembre 2017}

\begin{document}

\maketitle

\begin{abstract}
    À partir du \textit{jeu de la vie} de John Horton Conway, je crée des animations colorées et graineuses et considère la possibilité de créer une machine à générer un film d'animation complet.
\end{abstract}

%!TEX root = structure.tex

\section{Énoncé du problème}
\noindent Je veux construire une machine pouvant générer automatiquement une grande quantité de scènes animées que je pourrai monter moi-même par la suite. Je dois pouvoir visualiser les scènes en \textit{mode simple} avant d'en faire des exportations en \textit{mode graineux}, qui est immensément plus long à produire. Une scène est donc définie par un certain nombre de propriétés : 

\begin{enumerate}
\item Une échelle, définie dans le code par la variable \lstinline|gridScalar|. Cette propriété doit elle-même redéfinir les propriétés \lstinline|gridXAmount| et \lstinline|gridYAmount| (qui sont en fait des propriétés importantes d'une scène).
\item Une palette de couleur.
\item Une forme initiale (un ensemble de coordonnées $x$ et $y$).
\item Un nombre maximal d'images à produire.

\end{enumerate}

\newpage
\section{Génération et archivage de palettes de couleur}
\begin{lstlisting}
socket.on('pushJSONs', function(data) {
    JSONs = data;
});
socket.emit('pullJSONs', "");
\end{lstlisting}

\section{Formes initiales}
\noindent Un des importants objets qui me manquent est une structure de données pouvant contenir des formes qui initialisent le \textit{jeu de la vie}. Une \textit{forme initiale} doit être un ensemble de coordonnées $x$ et $y$, et cet ensemble doit être lui-même formé par un ensemble de formes \textit{primitives}. Une machine à créer des formes initiales doit pouvoir combiner ces formes primitives en agencements originaux, agencements qui dépendent et répondent aux variables \lstinline|gridXAmount| et \lstinline|gridYAmount|. Pour l'instant, j'ai défini une telle forme de cette façon :
\begin{lstlisting}
let wX = 30; // Horizontal offset where the line begins and ends.
let wY = 10; // Vertical offset where the line begins and ends.
let x = wX;
let y = wY;
// This traces a vertical line with wY amount of padding.
for (let i = 0; i < gridYAmount - wY * 2; i++) {
    setGridValue(x, y, 1);
    y++;
}
\end{lstlisting}
\noindent Il y aura clairement deux types de formes primitives : des formes qui s'adaptent aux dimensions de la grille et des formes qui en sont indépendantes.

\end{document}