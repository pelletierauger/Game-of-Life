%!TEX root = structure.tex

\section{Énoncé du problème}
\noindent Je veux construire une machine pouvant générer automatiquement une grande quantité de scènes animées que je pourrai monter moi-même par la suite. Je dois pouvoir visualiser les scènes en \textit{mode simple} avant d'en faire des exportations en \textit{mode graineux}, qui est immensément plus long à produire. Une scène est donc définie par un certain nombre de propriétés : 

\begin{enumerate}
\item Une échelle, définie dans le code par la variable \lstinline|gridScalar|. Cette propriété doit elle-même redéfinir les propriétés \lstinline|gridXAmount| et \lstinline|gridYAmount| (qui sont en fait des propriétés importantes d'une scène).
\item Une palette de couleur.
\item Une forme initiale (un ensemble de coordonnées $x$ et $y$).
\item Un nombre maximal d'images à produire.

\end{enumerate}

\newpage
\section{Structure du programme}
\noindent Je dois penser à la structure du programme avant tout le reste. Premièrement, puisque je \textit{pull} des données JSON sur le serveur, je dois créer un programme qui attend l'arrivée de ces données avant de démarrer le film. Ensuite, je dois créer un système qui initialise une scène. En fait, je dois avoir des x-sheets, aussi. Des x-sheets qui diffèrent considérablement de celles des \textit{Joies confuses}.

\subsection{Modes}
\noindent Premièrement, j'ai deux modes différents d'affichage. Comment les nommer? \lstinline|simple| et \lstinline|grainy|?

\subsection{Initialiser la grille}
\begin{lstlisting}
// This variable will hold the JSON assets when there are pushed by the server.
let JSONs;
// This variable will hold the scene when it's ready.
let scene;
// This variable will indicate that the grid is initialized.
let gridInitialized = false;

function draw() {
    if (JSONs) {
        scene.initializeGrid();
    }
    if (scene.gridInitialized) {
        scene.runGrid();
    }
}

function initializeGrid() {
    gridInitialized = true;
} 

\end{lstlisting}

\newpage
\section{Génération et archivage de palettes de couleur}
\begin{lstlisting}
socket.on('pushJSONs', function(data) {
    JSONs = data;
});
socket.emit('pullJSONs', "");
\end{lstlisting}

\section{Formes initiales}
\noindent Un des importants objets qui me manquent est une structure de données pouvant contenir des formes qui initialisent le \textit{jeu de la vie}. Une \textit{forme initiale} doit être un ensemble de coordonnées $x$ et $y$, et cet ensemble doit être lui-même formé par un ensemble de formes \textit{primitives}. Une machine à créer des formes initiales doit pouvoir combiner ces formes primitives en agencements originaux, agencements qui dépendent et répondent aux variables \lstinline|gridXAmount| et \lstinline|gridYAmount|. Pour l'instant, j'ai défini une telle forme de cette façon :
\begin{lstlisting}
let wX = 30; // Horizontal offset where the line begins and ends.
let wY = 10; // Vertical offset where the line begins and ends.
let x = wX;
let y = wY;
// This traces a vertical line with wY amount of padding.
for (let i = 0; i < gridYAmount - wY * 2; i++) {
    setGridValue(x, y, 1);
    y++;
}
\end{lstlisting}
\noindent Il y aura clairement deux types de formes primitives : des formes qui s'adaptent aux dimensions de la grille et des formes qui en sont indépendantes.

\newpage
\section{Idées errantes}
\begin{enumerate}
\item Je pourrais avoir des scènes qui seraient définies comme étant plus rapides que d'autres. 
\item Je pourrais avoir un système pour sauvegarder une \lstinline|shape| et la réutiliser par la suite, la connecter à une scène. 
\item Je pourrais avoir un \textit{offset} pour les valeurs dans l'array \lstinline|changes| qui enverrait une valeur différente aux fonctions de colorisation, puisque certaines palettes sont particulièrements belles à différents moments.
\end{enumerate}

\newpage
\section{Le flot dessinateur de la fonction \lstinline|draw()|}
\noindent L'impression d'une image démarre avec le dessin d'un \lstinline|background| noir. Ensuite, le \textit{flot dessinateur} cherche dans la grille la prochaine case dont la valeur est de 1. Il imprime une case qui est de valeur 1 à chaque \textit{frame}. Lorsqu'il a imprimé chaque case de la grille, il exporte l'image, incrémente la variable \lstinline|frameToExport|, renouvelle la grille et donne à la booléenne \lstinline|printedBackground| la valeur \lstinline|false|.

La nouvelle grille est ensuite imprimée de la même façon. \lstinline|frameToExport| ne sert qu'à nommer le fichier PNG qui est sauvegardé.
\begin{lstlisting}
if (boxToPrint >= grid.length - 1 && exporting == true) {
    console.log("This is working");
    boxToPrint = 0;
    frameExport(frameToExport);
    frameToExport++;
    updateGrid();
    printedBackground = false;
}
\end{lstlisting}Maintenant, je dois changer \lstinline|updateGrid()| pour \lstinline|updateScene()|, parce qu'une scène va réagir de façon différente. Une scène possède une propriété \lstinline|counter| et ne va mettre à jour sa grille que lorsque cette propriété répond à un test. De cette façon, je pourrai créer des scènes lentes dont les images graineuses vont vaciller.

En fait, tout ce que j'ai besoin de faire c'est d'encapsuler le \textit{flot dessinateur} dans un système de scènes. Ces scènes doivent pouvoir avoir des vitesses différentes.

\newpage
\section{Un système plus élégant et fonctionnel}
Présentement, je définis une scène ainsi : la quantité de colonnes de la grille est un multiple de 16, et la quantité de rangées de la grille est un multiple de 9. Ces deux valeurs sont multipliées par la propriété \lstinline|gridScalar| d'une instance de \lstinline|Scene|. Dans un système idéal, je devrais pouvoir définir une grille de n'importe quelle taille, définir un \lstinline|xOffset|, un \lstinline|yOffset| et un \lstinline|gridScalar| et ensuite afficher un sous-ensemble de la grille. Et ce sous-ensemble aurait des dimensions dans le format 16 : 9.