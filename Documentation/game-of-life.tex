%!TEX root = structure.tex

\section{Énoncé du problème}
\noindent Je veux construire une machine pouvant générer automatiquement une grande quantité de scènes animées que je pourrai monter moi-même par la suite. Je dois pouvoir visualiser les scènes en \textit{mode simple} avant d'en faire des exportations en \textit{mode graineux}, qui est immensément plus long à produire. Une scène est donc définie par un certain nombre de propriétés : 

\begin{enumerate}
\item Une échelle, définie dans le code par la variable \lstinline|gridScalar|. Cette propriété doit elle-même redéfinir les propriétés \lstinline|gridXAmount| et \lstinline|gridYAmount| (qui sont en fait des propriétés importantes d'une scène).
\item Une palette de couleur.
\item Une forme initiale (un ensemble de coordonnées $x$ et $y$).
\item Un nombre maximal d'images à produire.

\end{enumerate}

\section{Formes initiales}
\noindent Un des importants objets qui me manquent est une structure de données pouvant contenir des formes qui initialisent le \textit{jeu de la vie}. Une \textit{forme initiale} doit être un ensemble de coordonnées $x$ et $y$, et cet ensemble doit être lui-même formé par un ensemble de formes \textit{primitives}. Une machine à créer des formes initiales doit pouvoir combiner ces formes primitives en agencements originaux, agencements qui dépendent et répondent aux variables \lstinline|gridXAmount| et \lstinline|gridYAmount|. Pour l'instant, j'ai défini une telle forme de cette façon :
\begin{lstlisting}
let wX = 30; // Horizontal offset where the line begins and ends.
let wY = 10; // Vertical offset where the line begins and ends.
let x = wX;
let y = wY;
// This traces a vertical line with wY amount of padding.
for (let i = 0; i < gridYAmount - wY * 2; i++) {
    setGridValue(x, y, 1);
    y++;
}
\end{lstlisting}
\noindent Il y aura clairement deux types de formes primitives : des formes qui s'adaptent aux dimensions de la grille et des formes qui en sont indépendantes.